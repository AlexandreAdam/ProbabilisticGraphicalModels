\documentclass{article}
\usepackage[a4paper, margin=2cm]{geometry}

\usepackage{amsmath}
\usepackage{bm}
\usepackage{amstext}
\usepackage{amsthm}
\usepackage{fancyhdr}
\usepackage{amssymb}

\pagestyle{fancy}
\rhead{Alexandre Adam}
\lhead{IFT6269 ― Probabilistic Graphical Models}
\chead{Homework 1}
\rfoot{\today}
\cfoot{\thepage}


\newcommand{\s}{\hspace{0.1cm}}
\numberwithin{equation}{section}
\renewcommand\thesubsection{\alph{subsection})}
\renewcommand\thesubsubsection{\roman{subsubsection}}

\begin{document}

\section{Probability and independence}
\subsection{Decomposition}
We aim to validate
\begin{equation}
        (X \perp Y, W \mid  Z)  \implies (X \perp Y \mid Z)
\end{equation}
\textit{Proof}.  We suppose the statement $(X \perp Y,\ W \mid Z)$ is true. 
It follows from the definition of the conditional independence that
$p(x, y, w \mid z) = p(x\mid z) p(y, w\mid z)$ for all 
$x \in  \Omega_x$, $(y, w) \in \Omega_y \times  \Omega_w$  and $z \in  \Omega_z$.
We then consider the marginalize  $p(x,y,w \mid z)$:
\begin{align*}
        p(x,y|z) &=  \sum_{w\s \in \s \Omega_w}  p(x, y , w \mid z) \\
                 &=  \sum_{w\s \in \s \Omega_w} p(x\mid z)p(y,w \mid z) \\
                 &= p(x \mid z) \sum_{w\s  \in \s  \Omega_w} p(y,w \mid z) \\
                &=  p(x\mid z) p(y\mid  z)
\end{align*}
from which we conclude that  $(X \perp Y \mid Z) \qed$.
By symmetry of the argument,  we can show that ${(X \perp W \mid Z)}$ is true as well.

\subsection{}
We aim to validate 
\begin{equation}\label{eq:1b}
       (X \perp Y \mid Z) \s \text{and} \s (X, Y \perp  W \mid Z) \implies (X \perp W \mid Z)  
\end{equation}
\textit{Proof}. Suppose $(X, Y \perp  W \mid Z)$ and $(X \perp Y \mid Z)$ are true.
We know from the symmetry and decomposition properties of  
the  conditional independence that  
$(X, Y \perp W \mid  Z)  \implies  (W \perp X, Y \mid Z)  \implies  (X \perp  W \mid Z)$.
Therefore $ (X \perp W \mid Z)$ is true $\qed$.


\subsection{}
We aim to validate
\begin{equation}
        (X \perp Y,W \mid Z) \s \text{and} \s (Y \perp W \mid Z) \implies  (X,W \perp Y \mid Z)
\end{equation}
\textit{Proof}. Suppose $(X \perp Y,W \mid Z)$ is true. 
Then it follows from the definition of conditional independence that
\begin{align*}
        p(x,y,w \mid z) = p(x \mid z) p(y,w \mid z) 
\end{align*}
Then assume $(Y \perp W \mid Z)$ is true. The second factor can be factorized
\begin{align*}
        p(x,y,w \mid z) = p(x \mid z) p(y \mid z) p(w \mid z)
\end{align*}
From the decomposition property, we know $(X \perp W \mid Z)$ is true. Thus
\begin{align*}
        p(x,y,w \mid z)  = p(x,w \mid z) p(y \mid z)
\end{align*}
From which we conclude $(X,W \perp Y \mid Z)$ is true $\qed$.

\subsection{}
We aim to validate
\begin{equation}
        (X \perp Y \mid Z) \s \text{and} \s (X \perp Y \mid W) \implies (X \perp Y \mid Z,W)
\end{equation}
\textit{Counter example}. We consider the following R.V.
\begin{enumerate}
        \item X: Person A arrive late for diner;
        \item Y: Person B arrive late for diner;
        \item W: They come from the same city;
        \item Z: They work in the same city.
\end{enumerate}
For this situation, we see that X and Y are conditionally independent when 
given either W or Z. If we know they are from the same city, then they might work 
in different cities and take different route home. Thus knowing person A was late doesn't inform 
us on the probability of person B to arrive late. \par
A similar argument can be made for  $(X \perp Y \mid Z)$. \par
Thus the LHS of the proposition is true, yet the RHS is clearly false in our case. 
Assuming we were given that W and Z are true, then we are given the geolocalisation 
of person A and B.
If we were given that 
person A would be late for diner, then we'd be able to make a good guess that person B would 
be late as well (they would both be impacted by the same traffic jam or whatnot).
Thus the proposition is false.

\section{Bayesian inferance and MAP}
Let $\mathbf{X}_1, \dots , \mathbf{X}_n \bm{\mid} \bm{\pi} \overset{\text{iid} }{\sim} 
\text{Multinomial} (1, \bm{ \pi} )$ on k element. The values are sampled from a set of cardinality 
2, that is $x^{(i)}_j \in \left\{ 0, 1 \right\}$. Each R.V. has only one non-zero 
entry for a given trial, that is $\sum_{j = 1}^k x^{(i)}_j = 1$. \par
We assume a Dirichlet prior $\bm{ \pi} \sim \text{Dir} ( \bm{ \alpha} )$ with 
a PDF
\[
        p(\bm{ \pi}  \mid \bm{ \alpha} ) = \frac{\Gamma ( \sum _{i = 1}^k \alpha_j) }{
        \prod_{j=1}^k \Gamma(\alpha_j)} \prod_{j = 1}^k \pi_j^{\alpha_j - 1}
\]

\subsection{}
Since the data is IID, they are mutually independent by definition. Being given the 
parameters of their Multinomial distribution (or a subset for that matter) 
does not change the independence of the $\mathbf{X}$'s. Thus, 
\[
        (\mathbf{X}_i \perp \mathbf{X}_j \mid \bm{ \pi}) \s \forall \s (i,j) \s \in 
        \left\{ 1,\dots ,k \right\} \times \left\{ 1,\dots ,k \right\}
\]
Of course, none of the vector can be mutually nor conditionally independent to $\bm{ \pi} $ 
since it contains information about the distribution of the one hot vectors $\mathbf{X}_i$. 
In this case $\bm{ \pi} $ are the probabilities of one of the $k$ entry to be equal 
to one. Even giving one of these away is enough to impact the posterior distribution 
of the conditional $p(x_i\mid x_\ell, \pi_j )$ for example.

\subsection{}
The posterior distribution $p( \bm{ \pi} \mid x_1 , \dots , x_n) $ is computed via 
the Bayes rule
\[
        p(\bm{ \pi} \mid \mathbf{x}_{1:n}) = 
        \frac{p(\mathbf{x}_{1:n} \mid \bm{ \pi} ) p(\bm{ \pi}) }{p(\mathbf{x}_{1:n})}
\]
where $p(\bm{ \pi} ) = p(\bm{ \pi} \mid \bm{\alpha} )$ is the prior for $\bm{ \pi} $
defined above. For the sakes of determining the 
posterior distribution nature, we can postpone the derivation of the marginal likelihood.
The likelihood 
is given by the product of $n$ Multinomial PMF given individually by
\[
        p(\mathbf{x}_i \mid \bm{ \pi} ) = \frac{1}{\prod_{\ell=1}^k x^{(i)}_\ell!} 
        \prod_{j=1}^k \pi_j^{x^{(i)}_j}
\]
Since $\sum_{j=1}^kx^{(i)}_j = 1$, then the product in the denominator is unity.
Therefore, the posterior must be
\[
        p(\bm{ \pi} \mid \mathbf{x}_{1:n} ) \propto 
        \prod_{i = 1}^n \prod_{j=1}^k \pi_j^{x_j^{(i)}}
        \left( \frac{\Gamma(\sum_{\ell = 1}^k \alpha_\ell)}{\prod_{\ell}^k \Gamma(\alpha_{\ell})}
                \prod_{\ell = 1}^k \pi_\ell^{\alpha_\ell -1}
        \right) 
\]
We use the fact that we can swap around product operator for real numbers and ignore 
the Dirichlet normalizing constant for the sake of the argument
\[
         p(\bm{ \pi} \mid \mathbf{x}_{1:n} ) \propto 
        \prod_{i = 1}^n \prod_{j=1}^k 
                      \prod_{\ell = 1}^k \pi_\ell^{\alpha_\ell -1}\pi_j^{x_j^{(i)}}

\]
We can readily see that the resulting distribution will be a Dirichlet with 
updated $\alpha_\ell$'s. 

\textbf{The posterior distribution is a \underline{Dirichlet} distribution with parameters 
$\alpha_j'= \alpha_j + \sum_{i = 1}^n x_j^{(i)} $}.

\subsection{Marginal Likelihood}
 The marginal likelihood $p(\mathbf{x}_{1:n})$ 
is a normalizing constant defined as the 
integral of the numerator over all instantiation of $\bm{ \pi} $
\[
        p(\mathbf{x}_{1:n}) = \int_{\bm{ \Delta_k}} 
        p(\mathbf{x}_{1:n} \mid \bm{ \pi})  p(\bm{ \pi} )d^{(k)}\bm{ \pi}  
\]
where $\bm{ \Delta}_k $ is the probability simplex. In term of the quantities 
defined above, this is
\[
         p(\mathbf{x}_{1:n}) = \int_{\bm{ \Delta_k}} 
         d^{(k)}\bm{ \pi} 
        \prod_{i = 1}^n \prod_{j = 1}^k \pi_j^{x_j^{(i)}}
        \left( 
         \frac{\Gamma ( \sum _{\ell = 1}^k \alpha_\ell) }{\prod_{\ell=1}^k \Gamma(\alpha_\ell)} 
         \prod_{\ell = 1}^k \pi_\ell^{\alpha_\ell - 1}
        \right) 
\]
The $\pi_j$'s are independent variables since the simplex $\bm{ \Delta}_k $ is crucially 
defined as an affine plane in an Euclidian space which is supported by 
a set of orthonormal vectors.
%The crucial point is that the base space is an Euclidian space 
%spaned by orthonormal vectors.
Thus, our task is to evaluate $k$ identical integrals of the form
\[
        \int_0^1 d\pi_j \prod_{i = 1}^n \pi_j^{x_j^{(i)} + \alpha_j - 1}
        = 
        \left( \sum_{i = 1}^n x_j^{(i)} + \alpha_j \right)^{-1}
\]
Thus
\[
        \boxed{p(\mathbf{x}_{1:n}) = 
         \frac{\Gamma ( \sum _{\ell = 1}^k \alpha_\ell) }{\prod_{\ell=1}^k \Gamma(\alpha_\ell)} 
 \left( \sum_{i = 1}^n x_j^{(i)} + \alpha_j \right)^{-k}}
\]


\end{document}
